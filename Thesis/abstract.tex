\thispagestyle{plain}


    
    \justifying

             
    Environmental pollution as a result of continued fossil fuel usage has long been a bone of contention especially in the energy industry, particularly around transportation fuels. Transportation fuels, precisely jet fuels have long been studied to curb the production of harmful byproducts from the combustion of these fuels by proposing new jet fuel synthesis. Researchers have invested resources to the synthesis of alternative jet fuels that promise lower emissions of harmful gases such as NOx, SOx and particulates. Studies and research have yielded in the fruition of a catalytic process that involves the conversion of already existing gas-to-liquid fuels via the Fischer-Tropsch (F-T) synthesis for jet fuels that promise lower emissions. As a result of these strides, the oxidation of these alternative jet fuels is pertinent especially for new engine designs. Surrogate modeling of the F-T GTL fuels is proposed based on spectrometry analysis to determine the major components of the F-T GTL fuel.
    
    This study is done on the automated chemical kinetic mechanism generation and validated analysis of a three-component F-T GTL fuel surrogate chemical kinetic model, comprising of three primary reference fuels (PRFs); n-decane, iso-octane and n-propyl-cyclohexane. The detailed chemical kinetic mechanism of the three-component fuel blend is constructed using the open-source software for automated chemical kinetic mechanism generation, Reaction Mechanism Generator (RMG). Comparison between the concatenation of the PRFs was compared to a kinetic model that consists of the direct generation of the PRF species when lumped together from the onset for cross-reactions in RMG. 
    
    Further analysis and model validation was carried out using an open-source software for solving chemical kinetic, thermodynamic and transport problems, Cantera. The chemical kinetic models are analyzed and compare the auto-ignition characteristics (ignition delay time) to that of experimental measurements of the GTL fuels at a wide range of conditions as well as to other chemical kinetic models of the PRFs before concatenation and the GTL lumped fuel surrogate model. The computational comparison is made for the best framework in constructing a detailed GTL fuel surrogate chemical kinetic model as well as which approach is best suited to closely match experimental validation.
