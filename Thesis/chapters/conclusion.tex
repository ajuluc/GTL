\chapter{Conclusion and Future Work}


This chapter is a summary of this thesis work. What was proposed, done and what further work is needed to capture the effect of understanding the use of surrogate models in the analysis of Fischer-Tropsch (F-T) gas-to-liquid (GTL) fuel blends.

This work centered on the use of synthetic alternatives to the conventional-oil derived fuels in heavy duty and gas turbine applications, especially aeroderivative-gas turbine fuels - jet fuels. As regards this sustainable and alternative derivatives for jet fuels, the Fischer-Tropsch (F-T) synthesis process was seen to be a viable and interesting option especially since it has been in vogue since 1925. Since the original use of converting gas to liquid fuels was done for diesel engines, research from the literature has shown its viability in jet fuels \cite{Hermann2006ChemicalFuel} as well as  HCCI engines \cite{Mati2007TheModeling}. As a result, interest in the technology for processing and synthesising F-T GTL fuels has grown over the years, especially with the advent of renewable energy technologies in a quest to reduce carbon footprint and emissions level. 

The major analyses of F-T GTL blends has come from the spectrometry/chromatography analysis to determine the major constituents of these F-T GTL fuel blends. As such, major compounds include the presence of n-alkanes (n-paraffins), iso-alkanes (iso-paraffins), cyclo-alkanes (naphthenes) and sometimes aromatics. However, since aromatics have been wildly known to be responsible for the onset of soot and particulate matter (PM) in engines \cite{RoleOSTI.GOV}, these aromatics have been jettisoned for cyclo-alkanes (napthenes) with the knowledge that oxidation of naphthenes at low temperatures do produce aromatic compounds \cite{Abbasi2018KineticFormation}.

The use of chemical kinetic mechanisms to represent the behavior of these F-T GTL fuel blends has grown increasingly over the last years and since more work in being done to accelerate the growth of software technologies to match up with the demand of numerical and computational analysis of elementary reactions, especially in the reactivity and flammability of aero-propulsion engines, the role of computational combustion continues to increase. Engine designers will continue to rely heavily on the chemical kinetic mechanisms of different jet fuels to aid in CFD analyses of aero-engines. The use of direct numerical simulations (DNS) always proves very difficult and computationally expensive, thus it has become increasingly common to use a representation of a few chemical kinetic species to represent the kinetics of the overall fuel. This approach of representing complex chemical reaction kinetic networks with a simplified form is adopted in surrogate modeling in many diverse engineering applications. The use of these surrogate models is heavily relied on in chemical kinetic mechanisms of different fuels and F-T GTL fuel blends are no different in this regard. 

This thesis has employed a surrogate model for the representation of a F-T GTL fuel blend using a three component fuel of n-decane, iso-octane and n-propyl-cyclohexane in different compositions as seen in table \ref{fig:GTL-composition}. The use of these Primary Reference Fuels (PRFs) was been delved into in chapters 3, 4 and 5 including the high temperature and low temperature oxidation pathways for n-decane and iso-octane as well as the high temperature pathways of n-propyl-cyclohexane. The thermo-kinetic parameters used in estimating the model builds of each PRF was done in the Reaction Mechanism Generator (RMG) \cite{Gao2016ReactionMechanisms} using a base mechanism of previously published models seen in table \ref{tab:PRF_table1}. The seed mechanisms which serve as a base mechanism for the detailed kinetics of each of the PRF models helped in predicting the oxidation pathways as well as the rate parameters used in evaluating the detailed mechanisms. 

This thesis was proposed on the best applicable framework in which F-T GTL fuel blends can be effectively constructed using a three-component blends of n-decane, iso-octane and n-propyl-cyclohexane. The effective frameworks each have their respective drawbacks. 


\section{Concatentation of PRFs }
The first framework of concatenation of the three PRFs resulted in first building individual detailed kinetic mechanisms for n-decane, iso-octane and n-propyl-cyclohexane and using the appropriate modeling strategies (code) to validate the individual mechanisms against published experimental data that quantifies the auto-ignition characteristics of the PRFs. The PRFs were all validated against shock-tube experiments that measured the ignition delay times in zero-dimensional homogeneous reactors while the model of n-decane has an additional validation experiment with the laminar premixed flame speeds. The PRFs have been observed to be in close agreement and within reasonable uncertainties with the experimental data. While there were deviations in both the high temperature and low temperature mechanisms, the pronounced and well visible negative temperature coefficient (NTC) behavior was seen in the experimental model validation of n-decane and iso-octane. This NTC behavior was been attributed to the onset and rapid production and consumption of the oxygenated radicals in the \ce{OH^.} radicals, \ce{Q^.OOH} radicals, the cyclic ethers \ce{Q-O-R}, peroxyradicals \ce{ROO^.}, hydroperoxy radicals \ce{Q^.OOH}, hydroperoxy-alkyl-peroxy radicals \ce{HOOQOO^.}, di-hydroperoxy-alkyl radicals \ce{HOOQ^.OOH} and finally keto-hydroperoxy radicals. Since these pathways contain lots of oxygen atoms, the flux diagrams show that these pathways are competing pathways but are ultimately more favorable at low temperatures due to the low activation energies. The PRFs being concatenated based on composition implies that the greatest composition of the PRFs, ultimately determines the behavior of the model blend in the low temperature region and any onset of the NTC regime. While this concatenation method is preferred as the detailed PRFs have been well validated and rate parameters well inspected through sensitivity analyses, the parallel computing costs can be immense especially since all three-component PRFs must run in parallel while the concatenation occurs at the end removing or marking reactions as duplicate or using the order of the concatenation of the PRFs for the kinetic estimates used. This often is the go to choice especially since dealing with reactions in smaller more detailed forms removes the need for allowing model generation for long running times adequate for cross reactions to occur in the mixed form of using PRF species from the onset.




\section{Use of PRF species from the onset for cross reactions}

The second framework is the use of allowing the elementary and individual species of each PRF to be lumped together from the onset for cross reactions to occur. While this framework is the most time consuming and computationally expensive in the sense that all species must be allowed to react with themselves for a given time interval, the time it take for all reaction networks and pathways to occur can sometimes be infinite if tolerance thresholds are imposed for very accurate model generation. Thus, this brings with it the added complexity of addressing the reaction pathways as well as the rate parameter estimates based on the reaction and reaction family. While this method is used usually allows for more complete reactions, the framework alone usually results in weeks or even months of continuous model generation. 

This approach has been used in chapter 6, where the cross reactions are allowed and the F-T GTL blended model is compared with the mechanisms of Dooley et al \cite{Dooley2010AProperties}, Naik et al. \cite{Naik2011DetailedFuels}, Dagaut et al. \cite{Dagaut2014} and the experiments of ignition delay times in shock tubes of Wang et al. \cite{Wang2012}. As seen in fig.\ref{fig:gtl-blend-idt}, it is evident that the model with cross-reactions does hold up well against the models of Dooley et al \cite{Dooley2010AProperties}, Naik et al. \cite{Naik2011DetailedFuels}, Dagaut et al. \cite{Dagaut2014}. The important question that remains unanswered is the calculation on the computational cost as well as the reaction flux diagram of this method. Is it worth the minimum three weeks of high-performance computing with thermo-kinetic estimates that are highly uncertain ? The numerous questions of this method abounds and must be looked at in detail in subsequent works.

\section{Future Work and Recommendations}

The question as regards which framework represents the best approach to build a F-T GTL fuel surrogate blend that is closely matched by the experimental conditions is one that has been answered. As can be noted that obtaining the models of the PRFs namely, n-decane, iso-octane and n-propyl-cyclohexane is the prefered method sine the validation of each PRF has been guaranteed to conform well to experimental conditions. The context for further work remains. The most important further work needed is the estimation of the thermo-kinetic parameters used in modeling the GTL fuel surrogates as well as which model blend approach closely matches experimental conditions. The best approach is the concatenation of all PRFs since the detailed chemical kinetic mechanisms closely resembles the experimental data. More work is needed first in the following ways:
\begin{enumerate}
\begin{enumerate}
    \item Estimation of thermo-kinetic parameters used in the model generation of the chemical kinetic mechanisms of the PRFs as well as the cross reacted blends.
    \item The uncertainty analysis of the thermo-kinetic parameters used and how close they are to the physical occurrence of the estimates. This is so collision violators (rates that are above the collision rates are not used)
    \item Computational and numerical efficiency of the code to reduce computational cost through parallel computing
    \item Model validation against laminar premixed flame speeds by model reduction techniques such as Sensitivity Analysis \cite{Rabitz1983SensitivityKinetics},\cite{Turanyi1990SensitivityApplications}, Directed Related Graph Method (DRG) \cite{Lu2005AReduction}, Directed Related Graph with Error Propagation (DRGEP)\cite{Pepiot-Desjardins2008AnMechanisms}, Directed Related Graph with Sensitivity-Analysis (DRGASA)\cite{Sankaran2007StructureFlame}\cite{Zheng2007Experimental13-butadiene}, Directed Related Graph with Error Propagation and Sensitivity Analysis (DRGEPSA) \cite{Niemeyer2010SkeletalAnalysis}, Computational Singular Perturbation \cite{Massias1999AnData}, Rate-Controlled-Reduced-Equilibrium methods (RCCE) \cite{Keck1971Rate-controlledMixtures}, which is done to reduce these kinetic mechanisms to only the core skeletal models of only a few hundred species with unimportant reactions and species removed for easy numerical computation of reactors and engines in CFD analysis that engine designers use.
\end{enumerate}
\end{enumerate}

